
\def \Subject {آزمون امنیتی نرم افزار \lr{OWASP Juice Shop}}
\def \Course {طراحی نرم افزار امن - مباحثی در علوم کامپیوتر}

\begin{center}
\vspace{.4cm}
{\bf {\Large \Subject}}\\
\vspace{16pt}
{\bf \Large \Course}
\end{center}
\clearpage

%\huge{\Subject}\\[1.5 cm]
\section{مقدمه}
هدف اصلی این مستند، ارزیابی امنیتی و کشف آسیب‌پذیری‌های نرم‌افزار \lr{OWASP Juice Shop} است. این نرم‌افزار یک سامانه فروشگاهی متن‌باز است که به طور خاص جهت تمرین و آموزش مباحث امنیت وب طراحی شده است.

در این پژوهش، مسیر تحلیل به شرح زیر طی خواهد شد:
\begin{enumerate}
    \item راه‌اندازی زیرساخت امن و ایزوله برای تحلیل.
    \item بررسی قابلیت‌ها و منطق کسب‌وکار (\lr{Business Logic}) سامانه.
    \item مدل‌سازی تهدیدات و شناسایی اهداف بالقوه مهاجم.
    \item اجرای آزمون‌های امنیتی و نفوذ.
    \item ارائه راهکارهای امنیتی و اقدامات متقابل.
\end{enumerate}
ابزار محوری مورد استفاده جهت انجام این ارزیابی‌ها، نرم‌افزار \lr{BurpSuite} می‌باشد.

\section{راه‌اندازی محیط آزمایشگاه}
جهت پیاده‌سازی محیط شبیه‌سازی (شامل سیستم کاربر، مهاجم و سرور هدف) به صورت امن و استاندارد، کلیه فایل‌های پیکربندی و اسکریپت‌های لازم در مخزن \lr{GitHub} زیر قرار داده شده است:

\lr{\url{https://github.com/ASDAFI/WebApplication-Security-Assessment.git}}

\subsection{نسخه‌های مورد استفاده}
برای مدیریت سرویس‌ها و ایجاد محیط‌های ایزوله از پلتفرم \lr{Docker} استفاده شده است. جهت تضمین صحت و تکرارپذیری آزمایش‌ها، نسخه‌های زیر به‌کار گرفته شده‌اند:
\begin{itemize}
    \item سامانه هدف: \lr{OWASP Juice Shop v19.1.1}
    \item ابزار تحلیل: \lr{BurpSuite Professional 2024.1.1.4}
\end{itemize}

\subsection{معماری آزمایشگاه تحلیل}
محیط تحلیل یک \lr{Container} سفارشی‌سازی شده است که علاوه بر ابزارهای امنیتی، به سرویس \lr{Remote Desktop Protocol (RDP)} مجهز شده است تا رابط گرافیکی (\lr{UI}) مناسبی را در اختیار تحلیل‌گر قرار دهد.
هدف اصلی از این معماری، ایجاد یک محیط کاملاً ایزوله است تا فعالیت‌های تست نفوذ هیچ‌گونه تثیری بر سیستم‌عامل میزبان نداشته باشد.

\subsection{پیکربندی سامانه \lr{OWASP Juice Shop}}
برای راه‌اندازی سامانه هدف، از تصویر (\lr{Image}) رسمی موجود در \lr{Docker Registry} به آدرس زیر استفاده شده است:

\lr{\url{https://hub.docker.com/r/bkimminich/juice-shop}}

\subsection{شبکه‌سازی ایزوله}
مدیریت شبکه و ارتباط بین کانتینرها توسط ابزار \lr{Docker Compose} انجام می‌شود. ویژگی‌های کلیدی این شبکه عبارتند از:
\begin{itemize}
    \item \textbf{ارتباط داخلی:} کانتینرها در یک شبکه مشترک قرار دارند و سامانه \lr{OWASP Juice Shop} از طریق نام دامنه \lr{\url{http://juice-shop}} (با استفاده از \lr{DNS} داخلی داکر) در دسترس سایر ماشین‌ها قرار دارد.
    \item \textbf{ایزولاسیون اینترنت:} این شبکه به گونه‌ای پیکربندی شده است که فاقد دسترسی به اینترنت خارجی باشد، که این امر امنیت محیط آزمایشگاه را تضمین می‌کند.
\end{itemize}

\subsection{تعدد ماشین‌ها در شبکه}
جهت شبیه‌سازی دقیق حملات لایه شبکه (نظیر \lr{ARP Spoofing} و \lr{MITM})، نیاز به تفکیک ماشین مهاجم و قربانی است. این نیازمندی با تعریف \lr{Instance}های مجزا در فایل \lr{docker-compose} محقق شده است؛ به طوری که یک کانتینر نقش کاربر عادی و کانتینر دیگر نقش مهاجم را در شبکه ایفا می‌کند.

%\centerline{\rule{13cm}{0.4pt}}
\setcounter{equation}{0}

\newpage


\section{\lr{OWASP Juice Shop}}
این سامانه در ظاهر یک فروشگاه آنلاین خرید آب‌میوه است.

\subsection{قابلیت‌های کاربران}
در این بخش، سطح دسترسی و قابلیت‌های کاربران بررسی می‌شود. هدف از این بخش، شناخت عمومی سطح حمله است و در بخش‌های بعدی به تحلیل عمیق‌تر پرداخته خواهد شد.

\subsubsection{کاربر مهمان (بدون احراز هویت)}

\begin{itemize}

\item 
مشاهده لیست محصولات، قیمت‌ها و نظرات ثبت شده برای هر محصول در صفحه اصلی.

\item
امکان ثبت بازخورد (\lr{Feedback}) به صورت ناشناس.

\item
مشاهده اطلاعات توسعه‌دهندگان و تصاویر مرتبط در صفحه \lr{About US}.

\item
مشاهده تصاویر بارگذاری شده توسط سایر کاربران در بخش \lr{PhotoWall}.

\item 
امکان ورود به حساب کاربری (در صورت داشتن حساب) و یا ورود از طریق حساب گوگل (\lr{OAuth}).

\item
قابلیت ثبت‌نام و ایجاد حساب کاربری جدید.

\end{itemize}

\subsubsection{کاربر احراز هویت شده (لاگین شده)}
کاربر مهمان پس از ورود موفق به حساب کاربری، به این سطح دسترسی ارتقاء می‌یابد و قابلیت‌های زیر را دارا خواهد بود:

\begin{itemize}

\item
افزودن کالا به سبد خرید به تعداد دلخواه.

\item
ثبت نظر برای کالاها (حتی بدون خرید محصول).

\item
ارسال بازخورد (\lr{Feedback}) با نام کاربری مشخص.

\item
ثبت شکایت (\lr{Complaint}) و بارگذاری مدارک مربوطه.

\item
استفاده از سیستم پشتیبانی آنلاین در بخش \lr{Support Chat}.

\item
اشتراک‌گذاری تصویر و متن (\lr{Caption}) در بخش \lr{PhotoWall}.

\item 
امکان ارتقای حساب کاربری به \lr{Deluxe Membership} جهت دریافت تخفیف‌ها و ارسال سریع‌تر.

\item
\textbf{مدیریت حساب کاربری (\lr{Account}):}
    \begin{itemize}
        \item تغییر تصویر پروفایل و نام کاربری.
        \item مشاهده تاریخچه سفارش‌ها (\lr{Order History}) و ثبت نظر برای سفارش‌های تکمیل شده.
        \item درخواست بازیافت جعبه‌ها از طریق بخش \lr{Recycle}.
        \item مدیریت آدرس‌های ذخیره شده در \lr{My saved addresses} (افزودن، حذف و ویرایش).
        \item مدیریت روش‌های پرداخت در \lr{My Payment Options} (افزودن یا حذف کارت بانکی).
    \end{itemize}

\item
خروج از حساب کاربری (\lr{Logout}).

\item
درخواست حذف کامل اطلاعات کاربری از طریق مسیر \lr{Privacy and Security > Data Erasure Request}.

\item 
فعال‌سازی احراز هویت دو مرحله‌ای (\lr{2FA}) با استفاده از \lr{Google Authenticator} در بخش تنظیمات امنیتی.

\item
مشاهده آخرین \lr{IP} وارد شده به حساب کاربری.

\end{itemize}


\subsection{اهداف و انگیزه‌های مهاجم}

\subsubsection{تخریب و اختلال در سرویس}
هدف مهاجم می‌تواند آسیب به تجربه کاربری و شهرت سامانه باشد. این حملات شامل موارد زیر است:
\begin{itemize}
    \item دستکاری (\lr{Modify}) یا حذف داده‌های موجود در سایت.
    \item ارسال هرزنامه (\lr{Spam}) در بخش نظرات و بازخوردها.
    \item تغییر قیمت محصولات یا تغییر محتوای نظرات کاربران.
    \item بارگذاری محتوای نامناسب یا مستهجن که می‌تواند منجر به چالش‌های قانونی برای صاحب کسب‌وکار شود.
\end{itemize}

\subsubsection{کلاهبرداری مالی}
با توجه به ماهیت فروشگاهی سامانه، یکی از اهداف اصلی مهاجم خرید کالا با قیمت کمتر، رایگان و یا استفاده از حساب سایر کاربران برای پرداخت است.

\subsubsection{سرقت اطلاعات حساس}
استخراج اطلاعاتی نظیر آدرس‌های سکونت، شماره تلفن و اطلاعات کارت‌های بانکی کاربران. این داده‌ها ارزش بالایی برای حملات مهندسی اجتماعی و فیشینگ دارند و افشای آن‌ها می‌تواند خطرات جانی و مالی برای کاربران به همراه داشته باشد.

\subsection{سطح دسترسی و پیش‌فرض‌های حمله}
فرض بر این است که زیرساخت پایه 
(\lr{Docker Container}) امن است (بحث \lr{Trusted Computing Base (TCB)}) و تمرکز صرفاً بر آسیب‌پذیری‌های لایه نرم‌افزار (\lr{Application Layer}) می‌باشد.

همچنین جهت شبیه‌سازی یک سناریوی واقعی (\lr{Enterprise})، رویکرد آزمون جعبه سیاه (\lr{Black-box}) انتخاب شده است؛ بدین معنا که فرض می‌شود مهاجم دسترسی اولیه به کد منبع (\lr{Source Code}) ندارد. با این حال، در بخش ارائه راهکارها، به کد منبع ارجاع داده خواهد شد.

\subsubsection{دسترسی به \lr{API}ها}
تمام کاربران (حتی بدون ورود) به مجموعه‌ای از \lr{API}های عمومی دسترسی دارند که می‌تواند نقطه ورودی برای حملات باشد.

\subsubsection{هم‌بندی شبکه (\lr{Network Topology})}
قرارگیری مهاجم در یک شبکه محلی (\lr{LAN}) مشترک با قربانی، امکان پیاده‌سازی حملاتی نظیر \lr{Man-In-The-Middle} را فراهم می‌کند.

\subsubsection{پورت‌های باز}
بررسی پورت‌های باز جهت شناسایی سرویس‌ها و تکنولوژی‌های مورد استفاده ضروری است. سرویس‌هایی که به اشتباه در معرض اینترنت (\lr{Exposed}) قرار گرفته‌اند، می‌توانند حفره امنیتی ایجاد کنند.

در این سناریو، تنها پورت \lr{3000} جهت دسترسی به سامانه باز است و سایر پورت‌ها مسدود می‌باشند.

\newpage


\section{آزمودن API ها}
برای این تحلیل 
scope 
مربوط به
BurpSuite
را به 
دامنه مروبط به سرویس محدود میکنیم تا از نبض ها و سایر درخواست ها حین تحلیل در امان بمانیم.

\color{red}{مواردی که نیاز به تحلیل عمقی تر دارند و جالب هستند به رنگ قرمز در میآیند.}

\color{black}

\begin{figure}[H]
    \centering
    \includegraphics[width=0.8\textwidth]{assets/scope.png} 
    \caption{تنظیم Scope}
\end{figure}

سپس proxy مربوط به BurpSuite را در مرورگر تنظیم میکنیم تا به عنوان
middleware
قرار بگیرد و درخواست ها را مشاهده کنیم.
برنامه ای برای Drop کردن درخواست ها یا کنترل دستی نداریم و فعلا ناظر هستیم.

\subsection{ثبت نظرات}
ابتدا این آسیب پذیری را در صفحه customerFeedback
مشاهده خواهیم کرد. هنگام فراخوانی، مشاهده میشود که در API مربوط
به captcha
پاسخ آن نیز گنجانده شده است.



\begin{figure}[H]
    \centering
    \includegraphics[width=0.8\textwidth]{assets/feedback_captcha.png} 
    \caption{درخواست captcha در صفحه Feedback}
\end{figure}

ثبت درخواست برای نظر جدید بدین شکل است.

\begin{figure}[H]
    \centering
    \includegraphics[width=0.8\textwidth]{assets/feedback_req.png} 
    \caption{ثبت درخواست برای نظر جدید}
\end{figure}

\newpage
موارد مورد توجه برای بررسی:

برای این بررسی، درخواست ها به تب Repeater .برده میشوند


\begin{itemize}
        \item آیا می‌توان با همان captchaId مجدد نظر ثبت کرد و captcha دیگری نگرفت؟

        \begin{figure}[H]
            \centering
            \includegraphics[width=0.8\textwidth]{assets/out_of_5.png} 
            \caption{درخواست با همان captchaId و rating منفی}
        \end{figure}

        \item آیا می‌توان در فیلد rating عددی خارج از بازه ۱ الی ۵ انتخاب کرد؟

        تصویر مورد قبل به این نیز پاسخ میدهد.

        
        \item اگر rating ای ثبت کنیم که overflow کند چه می‌شود؟

        \begin{figure}[H]
            \centering
            \includegraphics[width=0.8\textwidth]{assets/long_score.png} 
            \caption{درخواست با rating ای که احتمالا overflow رخ دهد}
        \end{figure}

        \newpage

        \item آیا می‌توان با نام دیگری نظر داد؟ زیرا به نظر نام فرد در پرانتز درون خود نظر واقع است.

        \begin{figure}[H]
            \centering
            \includegraphics[width=0.8\textwidth]{assets/other_person_feedback.png} 
            \caption{نظر دهی جای فرد دیگر}
        \end{figure}

        همچنین این نظر را در صفحه AboutUs
        میتوان یافت.

        \begin{figure}[H]
            \centering
            \includegraphics[width=0.8\textwidth]{assets/bijani_feedback.png} 
            \caption{نظر دهی جای فرد دیگر با عدد خارج از بازه}
        \end{figure}

        \item این نظرات در چه صفحه ای نمایش داده میشوند و front-end برای آن چه کار میکند؟
        به طور مثال اگر ۱ میلیون ستاره نمایش داده شود - ممکن است صفحه بسیار آسیب ببیند یا حتی هنگ کند.

        در عکس قبلی پاسخ به این سوال واقع است.
        همانطور که مشاهده میشود برای اعداد منفی یا مثبت بزرگ تر از ۵ undefined ثبت میشود.

        \newpage
        
        \item خارج از front-end
        این نظرات از دیتابیس و سپس از سمت سرور چگونه بازخوانی شده اند؟

        \begin{figure}[H]
            \centering
            \includegraphics[width=0.8\textwidth]{assets/req.png} 
            \caption{نظرات فراخوانی شده از سمت سرور}
        \end{figure}

        پاسخ ها در صفحه AboutUs بدین شکل فراخوانی میشوند.
        درست است درون درخواست ما جای anonymous آدرس فرد دیگری را جا زدیم، اما همچنان userId مربوط null است اما برای کاربران واقعی اینطور نیست.
                
        \color{red}
                همچنین اعداد rating به طور صحیحی بازیابی شده اند.
        اما rating های اعداد بسیار بزرگ فراخوانی نشده اند. به نظر که احتمالا از سمت سرور خطایی حین ذخیره سازی رخ داده است.

        
        \color{black}

        \item
        وجود RateLimiter حس نمیگردد و ممکن است حملات DoS رخ دهد.
        میتواند دیتابیس (دیسک) را پر کند و به عملکرد های داخلی نیز آسیب بزند.
        
    \end{itemize}





\subsubsection{چکیده}

\begin{itemize}
        \item امکان ثبت نظر جای فرد دیگر ( نمایان در UI ) وجود دارد.
        \item امکان ثبت پیاپی بدون درگیر شدن با چالش captcha وجود دارد.
        \item امکان Rate دادن با عدد خارج از بازه ۱ تا ۵ وجود دارد اما Front-End توانایی بازنمایی آن را ندارد.
        \item عدم وجود محدود کننده نرخ میتواند حملات DoS با آسیب های جدی تر را در پی داشته باشد.
    \end{itemize}


\subsection{نهایی کردن خرید}
حین نهایی کردن خرید، درخواست POST ای مشاهده میشود که میتوان آن را چند بار ثبت کرد و یک سفارش را چند بار در سبد قرار داد.

\begin{figure}[H]
            \centering
            \includegraphics[width=0.8\textwidth]{order_submit.png} 
            \caption{ثبت سفارش تکراری}
        \end{figure}

\begin{figure}[H]
    \centering
    \includegraphics[width=0.8\textwidth]{orders.png} 
    \caption{صفحه سفارش ها}
\end{figure}

\subsection{راهکار های امنیتی}

\begin{itemize}
        \item برای فیلد ها سمت سرور باید verifier ایجاد شود.
        \item باید از rate limiter استفاده شود.
        \item درخواست هایی که یکبار ثبت میشوند باید وضعیتشان سمت دیتابیس آپدیت شود و قبل از آپدیت بررسی شوند که قبلا تایید شده اند یا خیر.
        \item اشتباهی نظیر ارسال پاسخ captcha به سمت کاربر نباید رخ دهد.
        \item captcha باید زماندار و یکبار مصرف باشد. 
        
    \end{itemize}

\section{حملات SQLi}

\subsection{درز ورود}
 آیا آسیب پذیری SQLi در صفحه ورود وجود دارد؟
    برای آزمایش این مورد، از ' به عنوان نام کاربری در api مربوط
    به login استفاده میکنیم.

    \begin{figure}[H]
            \centering
            \includegraphics[width=0.8\textwidth]{assets/sqli1.png} 
            \caption{خروجی API مربوط به ورود به ازای نام کاربری '}
    \end{figure}

    این مورد آسیب پذیری را نشان میدهد و نام جدول مربوط و فیلد را مشخص میکند.

    همچنین نوع DBMS استفاده شده، SQLit است.
  
\subsection{بدست آوردن اطلاعات}
\subsubsection{احراز هویت}
حال پس از یافتن درز ورود، اولین ایده تلاش برای ورود است.
از آنجایی که query مربوط به Login را مشاهده کردیم به سادگی با پر کردن دو فیلد نام کاربری و رمز عبور همانند تصویر زیر، توکن احراز هویت را دریافت میکنیم.

\begin{figure}[H]
            \centering
            \includegraphics[width=0.8\textwidth]{assets/sqli_auth.png} 
            \caption{احراز هویت به کمک SQLi}
        \end{figure}


\subsubsection{جدول Users}

به کمک ایده مقابل ابتدا درخواستی که خروجی مد نظر را در بر دارد میزنیم.

\begin{figure}[H]
            \centering
            \includegraphics[width=0.8\textwidth]{assets/users_tbl.png} 
            \caption{تلاش برای یافتن اطلاعات همه کاربران}
        \end{figure}

خروجی مد نظر \lr{jwt encode} شده است و نیاز به بازگردانی دارد.
پس از بازگردانی بدین شکل خواهد بود.

\begin{figure}[H]
            \centering
            \includegraphics[width=0.8\textwidth]{assets/users_tbl_decoded.png} 
            \caption{اطلاعات رمزگشایی شده کاربران}
        \end{figure}

\newpage

\subsubsection{جداول}
به کمک کوئری زیر در فیلد نام کاربری میتوان به جداول دست پیدا کرد.

\begin{LTR}
\begin{verbatim}
' UNION SELECT 1, group_concat(tbl_name), 3, 4, 5, 6, 7, 8, 9 FROM sqlite_master --
\end{verbatim}
\end{LTR}

زیر درون sqlite تمام جداول در sqlite_master قرار دارند.

چالشی که وجود دارد تعداد ستون های جدول Users است. این چالش را با 
زیرا عملیات Union که دو جدول را با هم اجتماع میگیرد، نیاز به دو جدول با تعداد ستون های برابر دارد.

آزمون کوئری
زیر را از ۱ شروع میکنیم و جلو میرویم. آخرین جایی که به ارور های سمت دیتابیس بر نخوردیم، تعداد ستون های جدول است.

\begin{LTR}
\begin{verbatim}
' Order by 1 --
\end{verbatim}
\end{LTR}

بنا بر این آزمایش، تعداد ستون ها ۱۳ است.

پس از انجام این کار خروجی بدین شکل است.

\begin{figure}[H]
            \centering
            \includegraphics[width=0.8\textwidth]{assets/fail1.png} 
            \caption{دریافت نام تمام جداول}
        \end{figure}


علت شکست بالا چیست؟ بدین دلیل است که ستون مربوط به otpSecret توسط ما مقدار دهی شده است و کاربر فرضی با این مقدار در آن وجود دارد. به نظر مقدار پیشفرض خالی برای آن "" است. پس آن را در شماره ستون ۹ ام ( ستون مربوط به آن ) لحاظ میکنیم.


\begin{figure}[H]
            \centering
            \includegraphics[width=0.8\textwidth]{assets/totp_fix.png} 
            \caption{فیکس ستون totp برای استخراج همه جداول}
        \end{figure}

پس از رمزگشایی خروجی به نام جداول میرسیم.

\begin{figure}[H]
            \centering
            \includegraphics[width=0.8\textwidth]{assets/tables_names.png} 
            \caption{نام جداول concat شده در یکی از ستون های مربوط به Users}
        \end{figure}

بنا به این ترتیب نام جداول عبارت است از

\begin{LTR}
\begin{verbatim}
Users,Users,sqlite_sequence,Addresses,
Baskets,Products,BasketItems,BasketItems,
Captchas,Cards,Challenges,Complaints,Deliveries,
Feedbacks,Hints,ImageCaptchas,Memories,PrivacyRequests,
Quantities,Recycles,SecurityQuestions,SecurityAnswers,
SecurityAnswers,Wallets
\end{verbatim}
\end{LTR}

\subsection{بدست آوردن Schemaی جداول یافت شده}
یکی از موارد یافتن Schema یا Data-Definition-Layer که همان DDL است میباشد.
DDL اطلاعات دقیق تر نظیر ارتباطات را به ما ارائه نیز میدهد.

به کمک ایده زیر DDL مربوط به جدول Wallets را استخراج میکنیم.



\begin{figure}[H]
            \centering
            \includegraphics[width=0.8\textwidth]{assets/wallets_schema.png} 
            \caption{SQLi برای یافتن DDL}
        \end{figure}

پس از رمزگشایی خروجی به نام جداول میرسیم.

\begin{figure}[H]
            \centering
            \includegraphics[width=0.8\textwidth]{assets/wallets_schema_decoded.png} 
            \caption{DDL جدول Wallets}
        \end{figure}

بدین ترتیب DDL این جدول عبارت است از:

\begin{LTR}
\begin{verbatim}
CREATE TABLE `Wallets` (
    `UserId` INTEGER REFERENCES `Users` (`id`) ON DELETE NO ACTION ON UPDATE CASCADE, 
    `id` INTEGER PRIMARY KEY AUTOINCREMENT,
    `balance` INTEGER DEFAULT 0,
    `createdAt` DATETIME NOT NULL,
    `updatedAt` DATETIME NOT NULL
)
\end{verbatim}
\end{LTR}

\subsection{استخراج داده از جدول}
برای استخراج داده های مورد نظر میتوان از ایده زیر کمک گرفت.

\begin{figure}[H]
            \centering
            \includegraphics[width=0.8\textwidth]{assets/extract_val.png} 
            \caption{استخراج مقدار از جداول}
        \end{figure}

پس از رمزگشایی خروجی به نام جداول میرسیم.
این خروجی مقدار عدد درون کیف پول هر فرد را متناظر با شناسه آن در خود دارد.

\begin{figure}[H]
            \centering
            \includegraphics[width=0.8\textwidth]{assets/extract_val_decoded.png} 
            \caption{مقدایر پول درون کیف پول افراد}
        \end{figure}

\subsection{افزودن مبلغ به کیف پول}
پس از دستور Update در این محل، تغییری در مقدار موجود در کیف مشاهده نمیشود.
به نظر دلیل آن این است که دستور فقط query را اجرا میکند و چیزی را سمت دیتابیس commit نمیکند.

\subsection{درز افزودن آدرس}
پس از ورود به حساب ( اینجا پس از نفوذ admin )
درز دیگری باری modification
یافت میشود که آن افزودن آدرس است.

پس از آزمایش ها برای raise کردن error و تماشای SQLi
موردی یافت نمیشود که این آسیب پذیری را نشان دهد.

همچنین به نظر میرسد validation شماره موبایل سمت سرور تعبیه شده است که حس طراحی امن تر اینجا را میدهد.

\subsection{راهکار های امنیتی}
\begin{itemize}
    \item خطاهای سمت دیتابیس نباید به صورت خام برگردانده شوند. خطا ها باید در سمت سرور لاگ شوند و صرفا در دیتابیس شناسه مربوط به خطا مشخص باشد.
    \item 
    از قابلیت کامپایل و بعد جا گذاری query مربوط به DBMS ها باید استفاده کرد.
    در این قابلیت ابتدا کوئری به \lr{logical} یا \lr{physical plan}
    تبدیل خواهد شد و سپس پارامتر ها در آن جای خواهد گرفت. این بهترین شیوه مقابله است.
    \item
    دسترسی Read/Write را به نسبت کوئری های مختلف محدود کنیم.
    
\end{itemize}

\newpage


\section{حملات XSS}

یکی از فیلد هایی که به نظر آسیب پذیر میرسد فیلد جستجوی سایت است.

\begin{figure}[H]
            \centering
            \includegraphics[width=0.8\textwidth]{assets/sample_xss.png} 
            \caption{نمونه حمله xss}
        \end{figure}



\section{حمله \lr{ARP Spoofing}}
به دلیل HTTP بودن درخواست ها،
آن ها به صورت \lr{plain text}
مشخص هستند و مقادیر به راحتی در شبکه محلی مشخص است.

به راحتی میتوان نام کاربری افراد و درخواست های افراد را مشاهده کرد، توکن های مربوط به احراز هویت را برداشت و از آن استفاده کرد.

به کمک قابلیت Interpret برنامه Burpsute میتوان این کار را به سادگی انجام داد.

\section{نحوه مشارکت اعضای تیم}

کد های LaTEX مربوط به این ارائه را میتوانید از مخزن گیتهاب آن به آدرس

\lr{\url{https://github.com/ASDAFI/WebApplication-Security-Assessment.git}}

دریابید.

\membername{علی صدفی}
\begin{itemize}[noitemsep, topsep=0pt, label=\textcolor{gray}{$\bullet$}]
    \item راه‌اندازی آزمایشگاه و زیرساخت‌های لازم برای تحلیل
\end{itemize}

\membername{نرگس حسینی}
\begin{itemize}[noitemsep, topsep=0pt, label=\textcolor{gray}{$\bullet$}]
    \item بررسی قابلیت‌های کاربران و نقش‌های مختلف در سیستم
    \item شناسایی و تحلیل تهدیدها و خطرات احتمالی
\end{itemize}

\membername{فعالیت مشترک (تیمی)}
\begin{itemize}[noitemsep, topsep=0pt, label=\textcolor{myred}{$\diamond$}]
    \item ایده‌پردازی و آزمودن راهکارهای مختلف به صورت دو نفره ( \lr {Pair Work} )
\end{itemize}

